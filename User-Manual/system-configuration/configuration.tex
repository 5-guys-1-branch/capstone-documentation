\chapter{System Configuration}
\section{Object Recognition}

\begin{itemize}
\item \textbf{Darknet}

Further details on configuring Darknet can be found on the \href{https://github.com/AlexeyAB/darknet}{\textit{github page}} of the fork we used by AlexeyAB.

\item \textbf{Configure Makefile}

In the darknet\_ directory, locate the Makefile and change the following lines with reference to your system:

\begin{itemize}
    \item[\textbf{\#}] 89  - \textit{/path/to/cuda/include}
    \item[\textbf{\#}] 92 - \textit{/path/to/cuda/lib}
    \item[\textbf{\#}] 94 - \textit{/path/to/cuda/lib64}
    \item[\textbf{\#}] 101 - \textit{/path/to/usr/local/cuda/include}
    \item[\textbf{\#}] 102 - \textit{/path/to/usr/local/cuda/lib}
    \item[\textbf{\#}] 104 - \textit{/path/to/usr/local/cudnn/include}
    \item[\textbf{\#}] 105 - \textit{/path/to/usr/local/cudnn/lib64}
\end{itemize}

These lines must reflect the installation folder for cuda and cudnn specific to your device and it's installation folders.

\item \textbf{Configure config file for training}

In the darknet\_/cfg directory, get 300IQ, locate the animals.cfg and change the following lines:

\begin{itemize}
    \item Uncomment lines 3 - 6
    \item Commend out lines 8 - 11
    \item[\textbf{\#}] 21  - \textit{max\_batches = (classes * 2000)}
    \item[\textbf{\#}] 23 - \textit{steps = (classes*0.8), (classes*0.9)}
    \item[\textbf{\#}] 230 - \textit{filters = (classes + 5)*5}
    \item[\textbf{\#}] 236 - \textit{classes = no. classes}
    \item[\textbf{\#}] 250 - \textit{random = 0 or 1 to toggle random image editing during training}
\end{itemize}

* Note - Don't actually type in a formula like (classes * 2000). Only type in the value once you've calculated it

\end{itemize}

\section{Communication Server \& Jetson Nano}

\begin{itemize}
\item \textbf{Configure commserver.service file}

In the home directory on your jetson nano create a service file:

\begin{itemize}
    \item[\$] \textit{touch commserver.service}
\end{itemize}

Paste the following into the file: 

\fbox{\begin{minipage}{35em}
    [Unit]
    Description=Python Communication Server\newline
    After=network-online.target\newline
    \newline
    [Service]\newline
    Restart=on-failure\newline
    WorkingDirectory=/home/jetson/Follow-Me-Drones/server\newline
    ExecStart=/usr/bin/python3 /home/jetson/Follow-Me-Drones/server/comms.py\newline
    \newline
    [Install]\newline
    WantedBy=multi-user.target\newline
\end{minipage}}

Enable the system service and start it.

\begin{itemize}
    \item[\$] \textit{systemctl enable commserver.service}
    \item[\$] \textit{systemctl start commserver.service}
\end{itemize}

\end{itemize}